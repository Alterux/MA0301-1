\titlebox{Øistein Søvik - Øving 1}{}

\section*{Seksjon 2.1}

\begin{problem}[3] Let $p$, $q$ be the primitive statements for which the
  implication $p \to q$ false. Determine the truth values for each of the following:
  \begin{enumerate}
      The implication $p \to q$ is only false when $p$ is True and $q$ is
      False. Thus
  \addtocounter{enumii}{1}
    \item $\neg p \vee q$
  \end{enumerate}
  %
  Is False (or $0$)
  %
  \begin{enumerate}[resume]
  \addtocounter{enumii}{1}
    \item $\neg q \to \neg p$
  \end{enumerate}
  %
  This is just the same staten written as a contrapositive. $\neg q$ = True and
  $\neg p$ = False. Thus, the statement is False (or $0$).
\end{problem}
%
\begin{problem}[8]
  Construct a truth table for each of the following compounded statements, where
  $p$, $q$, $r$ denote primitive statements. 
  \begin{enumerate}
  \addtocounter{enumii}{3}
    \item $(p \to q) \to (q \to p)$
  \end{enumerate}
  %
  \begin{tabular}{L L L L L}
    \toprule
    p & q & (p \to q) & (q \to p) & (p \to q) \to (q \to p) \\
    \midrule
    0 & 0 & 1 & 1 & 1 \\
    0 & 1 & 1 & 0 & 0 \\
    1 & 0 & 0 & 1 & 1 \\
    1 & 1 & 1 & 1 & 1 \\
    \bottomrule
  \end{tabular}
  %
  \begin{enumerate}[resume]
    \item $[p \wedge (p \to q)] \to q$ 
  \end{enumerate}
  %
  \begin{tabular}{L L L L L}
    \toprule
    p & q & (p \to q) & [p\wedge(p\to q)] & [p \wedge (p \to q)] \to q \\
    \midrule
    0 & 0 & 1 & 0 & 1 \\
    0 & 1 & 1 & 0 & 1 \\
    1 & 0 & 0 & 0 & 1 \\
    1 & 1 & 1 & 1 & 1 \\
    \bottomrule
  \end{tabular}
  %
  \begin{enumerate}[resume]
    \item $(p \wedge q) \to p$ 
  \end{enumerate}
  %
  \begin{tabular}{L L L L}
    \toprule
    p & q & (p \wedge q) & (p \wedge q) \to p\\
    \midrule
    0 & 0 & 0 & 1 \\
    0 & 1 & 0 & 1 \\
    1 & 0 & 0 & 1 \\
    1 & 1 & 1 & 1 \\
    \bottomrule
  \end{tabular}
  %
  \begin{enumerate}[resume]
    \item $q \leftrightarrow (\neg p \vee \neg q)$ 
  \end{enumerate}
  %
  \begin{tabular}{L L L L}
    \toprule
    p & q & (\neg p \vee \neg q) & q \leftrightarrow (\neg p \vee \neg q) \\
    \midrule
    0 & 0 & 1 & 0 \\
    0 & 1 & 1 & 0 \\
    1 & 0 & 1 & 0 \\
    1 & 1 & 0 & 0 \\
    \bottomrule
  \end{tabular}
  %
  \begin{enumerate}[resume]
    \item $[(p \to q) \wedge (q \to r)] \to (p \to r)$
  \end{enumerate}
\end{problem}
%
  \begin{tabular}{L L L L L L L}
    \toprule
    p & q & r & (p \to q) & (p \to r) & (q \to r) & [(p \to q) \wedge (q \to r)] \to (p \to r)\\
    \midrule
    0 & 0 & 0 & 1 & 1 & 1 & 0 \\
    0 & 0 & 1 & 1 & 1 & 1 & 1 \\
    0 & 1 & 0 & 1 & 1 & 0 & 1 \\
    0 & 1 & 1 & 1 & 1 & 1 & 1 \\
    1 & 0 & 0 & 0 & 0 & 1 & 1 \\
    1 & 0 & 1 & 0 & 1 & 0 & 1 \\
    1 & 1 & 0 & 1 & 0 & 1 & 0 \\
    1 & 1 & 1 & 1 & 1 & 1 & 1 \\
    \bottomrule
  \end{tabular}
  %
\begin{problem}[15]
  The integer variables $m$ and $n$ are assigned the values $3$ and $8$,
  respectively, during the execution of a program. Each of the following
  \emph{successive} statements is then encountered during program execution.
  What are the values of $m$, $n$ after each of these statements are
  encountered?
  %
  \begin{enumerate}
    \item
      \begin{algorithmic}
        \STATE \IF{$n - m = 5$} $n := n - 2$ \ENDIF
      \end{algorithmic}
  \end{enumerate}
  %
  $m = 3$ and $n = 8  - 2 = 6$
  %
  \begin{enumerate}[resume]
    \item 
      \begin{algorithmic}
        \STATE \IF{$(2*m = n)$ \AND $(\lfloor n/4 \rfloor = 1)$}
        $n := 4 * m - 3$ \ENDIF
      \end{algorithmic}
  \end{enumerate}
  %
  $m = 3$ and $n = 4\cdot 3 - 3 = 9$.
  %
  \begin{enumerate}[resume]
    \item
      \begin{algorithmic}
        \STATE \IF{$(n < 8)$ \OR $(\lfloor m/2 \rfloor = 3)$}
              \STATE $n := 4 * m - 3$
        \ELSE
              \STATE $m = 2*n$
        \ENDIF
      \end{algorithmic}
  \end{enumerate}
  % 
  Neither of the conditions are true. Thus,  $m = 2\cdot 9 = 18$ and $n=9$.
  %
  \begin{enumerate}[resume]
    \item
      \begin{algorithmic}
        \STATE \IF{$(m < 20)$ \AND $(\lfloor n/6 \rfloor = 1)$}
        $m := m - n - 5$ \ENDIF
      \end{algorithmic}
  \end{enumerate}
  %
  Both conditions holds, thus $m = 18 - 9 - 5 = 4$ and $n = 9$.
  %
  \begin{enumerate}[resume]
    \item
      \begin{algorithmic}
        \STATE \IF{$((n = 2 * m)$ \OR $(\lfloor n/2\rfloor = 5))$}
        $n := 4 * m - 3$ \ENDIF
      \end{algorithmic}
  \end{enumerate}
  %
  Neither conditions holds, thus $m = 4$ and $n = 9$.
\end{problem}

\section*{Seksjon 2.2}

\begin{problem}[6]
  Negate each of the following and simplify the resulting statement.
  \begin{enumerate}
  \addtocounter{enumii}{1}
    \item $(p \wedge q) \to r$
  \end{enumerate}
  %
  \begin{align*}
    \neg [(p \wedge q) \to r]
      = p \wedge q \wedge \neg r 
  \end{align*}
  %
  As $\neg(a \implies b) = a \wedge \neg b$ I am not sure how to expand on the calculations..
  \begin{enumerate}[resume]
  \addtocounter{enumii}{1}
    \item $p \vee q \vee (\neg p \wedge \neg q \wedge r)$
  \end{enumerate}
  %
  \begin{align*}
        \neg \left[ p \vee q \vee (\neg p \wedge \neg q \wedge r)\right] 
    & = \neg p \wedge \neg q \wedge \neg (\neg p \wedge \neg q \wedge r) \\
    & = \neg p \wedge \neg q \wedge (p \vee q \vee \neg r)  \\
    & = \neg p \wedge \neg q \wedge \neg r
  \end{align*}
Used DeMorgans laws in the first equality and removed inverses in the second.
\end{problem}
%
%
\begin{problem}[7]
  \begin{enumerate}
    \item If $p$, $q$ are primitive statements, prove that
      %
      \begin{equation*}
        (\neg p \vee q) \wedge (p \wedge (p \wedge q)) \ \leftrightarrow \ (p \wedge q)
      \end{equation*}
  \end{enumerate}
  %
  \begin{align*}
                      (\neg p \vee q) \wedge (p \wedge (p \wedge q))
    \ \Leftrightarrow \ & (\neg p \vee q) \wedge ((p \wedge p) \wedge q) && \text{Commutative laws}\\ 
    \ \Leftrightarrow \ & (\neg p \vee q) \wedge (p \wedge q) && \text{Idempotent laws}\\ 
    \ \Leftrightarrow \ & [ (\neg p \wedge p) \vee (q \wedge p) ] \wedge q && \text{Distributive laws}\\
    \ \Leftrightarrow \ & [ F_0 \vee (q \wedge p) ] \wedge q && \text{Inverse laws}\\
    \ \Leftrightarrow \ & [ (q \wedge p) ] \wedge q && \text{Identity laws}\\
    \ \Leftrightarrow \ & p \wedge (q \wedge q)  && \text{Commutative laws}\\
    \ \Leftrightarrow \ &  (p \wedge q) && \text{Idempotent laws}\\
  \end{align*}
  %
  Alternatively the statement follows from the table below
  
  \begin{tabular}{L L L L L L L}
    \toprule
    p & q & (\neg p \vee q) & (p \wedge (p \wedge q)) &(\neg p \vee q) \wedge (p \wedge (p \wedge q)) & (p \wedge q)\\
    \midrule
    0 & 0 & 1 & 0 & 0 & 0 \\
    0 & 1 & 1 & 0 & 0 & 0 \\
    1 & 0 & 0 & 0 & 0 & 0 \\
    1 & 1 & 1 & 1 & 1 & 1 \\
    \bottomrule
  \end{tabular}
  %

  %
  \begin{enumerate}
      \item Write the dual of the logical equivalence in \textbf{a)}.
  \end{enumerate}
  The dual of a compound proposition that contains only the logical operators $\vee$
  , $\wedge$ , and $\neg$ is the compound proposition obtained by replacing each
  $\vee$ by $\wedge$
  , each $\wedge$ by $\vee$ , each $T_0$ by $F_0$ , and each $F_0$ by $T_0$ .
  %
  \begin{equation*}
    (\neg p \wedge q) \vee (p \wedge (p \wedge q)) \leftrightarrow (p \vee q)
  \end{equation*}
\end{problem}

\begin{problem}[18]
  Give the reasons for earch step in the following simplifications of compound
  statements
      \begin{flalign*}
 \textbf{\theenumii)}&&                   & [(p \vee q) \wedge (p \vee \neg q)] \vee q && \textbf{Reasons}\\
                     && \Leftrightarrow \ & [p \vee (q \wedge \neg q)] \vee q &&
                     \text{Idempotent \& Commutative \& Distributive laws}\\
                     && \Leftrightarrow \ & (p \vee F_0 ) \vee q &&
                     \text{Inverse laws}\\
                     && \Leftrightarrow \ & p \vee q && \text{Identity laws}
      \end{flalign*}
      
\end{problem}


%%% Local Variables:
%%% mode: latex
%%% TeX-master: "MA0301"
%%% End:
