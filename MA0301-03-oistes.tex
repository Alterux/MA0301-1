\documentclass[a4paper, english, 12pt]{article} % norsk & english is supported

\newcommand{\exerciseNumber}{3}
\newcommand{\solutions}{true} % Change to   true   to create solutions
% Must be defined before course package, \exerciseNumber, and \solutions
\usepackage{IMF}

\usepackage{MA0301}

\begin{document}

\titlebox

\begin{problem}[7]
  Use a truth table to show that $\bigl( (a \wedge b) \longrightarrow c \bigr)
  \Leftrightarrow
  \bigl( (a \longrightarrow c) \vee (b \longrightarrow c) \bigr)$.
\end{problem}

\begin{answer}
  \begin{table}[htbp!]
    \centering
    \begin{tabular}{C C C C C C C}
      \toprule
      a & b & c &  (a \to c) & (b \to c) & (a \to c) \vee (b \to c) & (a \wedge b) \to c \\
      \midrule
      \F & \F & \F & \T & \T & \T & \T \\
      \F & \F & \T & \T & \T & \T & \T \\
      \F & \T & \F & \T & \F & \T & \T \\
      \F & \T & \T & \T & \T & \T & \T \\
      \T & \F & \F & \F & \T & \T & \T \\
      \T & \F & \T & \T & \T & \T & \T \\
      \T & \T & \F & \F & \F & \F & \F \\
      \T & \T & \T & \T & \T & \T & \T \\
      \bottomrule
    \end{tabular}
  \end{table}
\end{answer}

\seksjon{3.1}

\begin{problem}[6]
  Consider the following six subsets of $\Z$.

  \begin{centering}
    \noindent \!\!\!\!
    \NumTabs{2}
  \begin{enumerate*}[itemjoin=\tab, label = $\Alph*$]
    \item \label{A} $= \{ 2m + 1 \mid m \in \Z \}$ 
    \item \label{B} $= \{ 2n + 3 \mid n \in \Z \}$
    \item \label{C} $= \{ 2p - 3 \mid p \in \Z \}$
    \item \label{D} $= \{ 2r + 1 \mid r \in \Z \}$ 
    \item \label{E} $= \{ 3s + 1 \mid s \in \Z \}$
    \item \label{F} $= \{ 3t - 2 \mid t \in \Z \}$
    \end{enumerate*}
  \end{centering}
  
  Which of the following statements are true, and which are false?
\end{problem}

\begin{centering}
  \NumTabs{3}
  \begin{enumerate*}[itemjoin=\tab, label = \textbf{\alph*)}]
  \item \ref{A} = \ref{B} \ans{\textbf{True}} 
  \item \ref{A} = \ref{C} \ans{\textbf{True}}
  \item \ref{B} = \ref{C} \ans{\textbf{True}}
  \item \ref{D} = \ref{E} \ans{\textbf{False}}
  \item \ref{D} = \ref{F} \ans{\textbf{True}}
  \item \ref{E} = \ref{F} \ans{\textbf{False}}
  \end{enumerate*}
\end{centering}

\ans{\newpage}

\seksjon{3.2}

\begin{problem}[6]
  Prove each of the following results without using Venn diagrams or membership tables.
\end{problem}

\begin{subproblem}[1]
  If $ A \subseteq B$ and $C \subseteq D$, then $A \cap C \subseteq B \cap D$ and $A
  \cup C \subseteq B \cup D$.
\end{subproblem}

\begin{answer}
  Assume $x \in A \cap C$, then $x \in A$ and $x \in C$. Now  $x \in A
  \Rightarrow x \in B$ since $A \subseteq B$, similarly $x \in C \Rightarrow x \in D$ since
  $C \subseteq D$. As $x \in B$ and $x \in D$ then $x \in B \cap D$. Thus $A
  \cap C \subseteq B \cap D$. To see that the reverse implication fails, choose
  $x$ such that $x \in (B / A) \cap (D / C)$ then $x \in B \cap D$, but $x \not
  \in A \cap C$. This proves $A \cap C \subseteq B \cap D$.

  Let $x \in A \cup C$. Then $x \in A \vee x \in C$. If $x \in A$ then $x \in B$
  since $A \subseteq B$, similarly if $x \in C$ then $x \in D$ since $C
  \subseteq D$. This shows that $x \in B \cup D$. For the reverse implication
  choose $ x \in (B / A) \cup (C / D)$, then $x \in B \cup D$, however $x
  \not\in A \cup C$. This proves $A \cup C \subseteq B \cup D$. 
\end{answer}

\begin{problem}
  Prove or disprove each of the following:
\end{problem}

\begin{subproblem}[2]
  For sets $A, B, C \subseteq \U$, $A \cup C = B \cup C \implies A = B$.
\end{subproblem}

\begin{answer}
  Let $A \neq B$ and $C = \U$, then $A \cup C = B \cup C$.\\
  More concretely let $\U = C = \N$, $A = \{p\}$, $B = \{q\}$, $p,q \in \N$. Then
  $A \cup C = B \cup C$, however $A \neq B$ when $p \neq q$. 
\end{answer}

\begin{subproblem}[4]
  For sets $A, B, C \subseteq \U$, $A\,\Delta\,C = B\,\Delta\,C \implies A
  = B$.
\end{subproblem}
%
\begin{answer}
  Assume that $x \in A$, then either $x$ lies in $C$ or not in $C$. If
  $x \in C$ then $x \not \in A \Delta C \Rightarrow \not \in B \Delta C
  \Rightarrow x \in B$. Else if $x \not\in C$ then $x \in A \Delta C \Rightarrow B
  \Delta C \Rightarrow x \in B$, as $x \not \in C$. Thus, $A \subseteq B$.


  The other direction is shown in precicely the same way. 
  Assume that $x \in B$, then either $x$ lies in $C$ or outside $C$. If $x \in
  C$, then $x \not \in B \Delta C \Rightarrow x \not \in A \Delta C \Rightarrow x
  \in A$. Else if $x \not \in C$ then $x \in B \Delta C \Rightarrow A \Delta C
  \Rightarrow x \in A$, as $x \not \in C$. Thus, $B \subseteq A$.

  As $A \subseteq B$ and $B \subseteq A$, then $A = B$, which is what was to be proven.
\end{answer}

\begin{problem}[16]
  Provide the justifications for the steps that are needed to simplify the set
  %
  \begin{equation*}
    (A \cap B) \cup [ B \cap ((C \cap D) \cup (C \cap \overline{D})) ]
  \end{equation*}
  %
  where $A, B, C, D \subseteq \U$.
  % 
  \begin{align*}
      & \textbf{Steps} && \textbf{Reasons} \\
      & (A \cap B) \cup [ B \cap ((C \cap D) \cup (C \cap \overline{D})) ]
      && \ans{} \\
    = & (A \cap B) \cup [B \cap (C \cap (D \cup \overline{D}))]
      && \ans{\text{Distributive Laws}} \\
    = & (A \cap B) \cup [B \cap (C \cap \U)]
      && \ans{\text{Inverse Laws}}\\
    = & (A \cap B) \cup (B \cap C)
      && \ans{\text{Domination Laws}}\\ 
    = & (B \cap A) \cup (B \cap C)
      && \ans{\text{Commutative Laws}}\\
    = & B \cap (A \cup C)
      && \ans{\text{Distributive Laws}}
\end{align*}
%
\end{problem}

\seksjon{2.5}

\begin{problem}[8]
  \begin{subproblem}
    \label{problem:2.5.8.a}
    Let $p(x)$, $q(x)$ be open statements in the variable $x$, with a given
    universe. Prove that
    % 
    \begin{equation*}
      \forall\,x \ p(x) \vee \forall \, x \ q(x)
      \implies
      \forall x [p(x) \vee q(x)]
    \end{equation*}
    % 
  \end{subproblem}
\end{problem}

\begin{answer}
  Universe $\U$.
  Assume that $\forall x \ p(x) \vee \forall
  x \ q(x)$ is true $x \in \U$. Assume that $\forall x \ p(x)$ is true, then
  there exists some $c \in \U$ such that $p(c)$ is true, and thus $p(c) \vee q(c)$ is true. Since $c$ can be
  choosen arbitarily, we have shown that $\forall x [p(x) \vee q(x)]$, which is
  what we wanted to show. If we instead assume that $\forall x \ q(x)$ is true,
  the exact same argument can be made.
\end{answer}

\begin{subproblem}
  Find a counterexample to the converse in part \ref{problem:2.5.8.a}. That is,
  find open statements $p(x)$, $q(x)$, and a universe such that $\forall x [p(x)
  \vee q(x)]$ is true, while $\forall\,x \ p(x) \vee \forall \, x \ q(x)$ is false.
\end{subproblem}

\begin{answer}
  Let $\U$ be some universe such that there exists non-empty disjunctive subsets $A$, $B$
  such that ($A \cap B = \emptyset$) and $\U = A \cup B$.

  Let $p(x)\ \colon x \in A$, and similarly $q(x)\ \colon x \in B$. If
  $y \in B $, then $p(y)$ is false (as $A \cap B = \emptyset$), thus $p(x)$ can
  not hold for all $x$ in other words $\forall x
  \ p(x)$ is false. Similarly, let $y \in A$ then $q(y)$ is false (again since
  $A \cap B = \emptyset$), thus $\forall x \ q(x)$ is false.
  However, $\forall x [p(x) \vee q(x)]$ is true as for every $x \in \U = A \cup
  B$. \medskip

  For a concrete example let $\U = \N$, $n \in \N$, and let $p(n) \colon n \text{ is
    odd}$, $q(n) \colon n \text{ is even}$. Then, $p(n)$ is not true for every
  $n$ as there exists even numbers, and similarly $q(n)$ is not true for every
  $n$ as there exists odd numbers. However, every $n$ is either
  even or odd. 
\end{answer}

\ans{\newpage}

\begin{problem}[10]
  Provide the missing reasons for the steps verifying the following argument:
  %
  \begin{align*}
    & \forall\,x\ [p(x) \vee q(x)]\\
    & \exists \,x\ \neg p(x)\\ 
    & \forall \,x\ [\neg q(x) \vee r(x)]\\
    & \forall \,x \ [s(x) \to \neg r(x)] \\[-0.5cm]
    \cmidrule{1-2} \\[-1cm]
     \therefore \ & \exists \, x \ \neg s(x)
  \end{align*}

  \begin{tabular}{r L p{1cm} p{8cm}}
    \textbf{Steps} & & & \textbf{Reasons} \\
    \ITEM \label{step-1}& \forall\,x\ [p(x) \vee q(x)]
        & & Premisse\\
    \ITEM \label{step-2} & \exists\,x\ \neg p(x)
        & & Premisse \\
    \ITEM \label{step-3} & \neg p(a)
        & & \Cref{step-2} and the definition of truth for $\exists \,x\ p(x)$. The reason for this step is also referred to as the \emph{Rule of Existential Specification}\\
    \ITEM \label{step-4} & p(a) \vee q(a)
        & & \ans{ \Cref{step-1} and the \emph{Rule of Universal Specification}}\\
    \ITEM \label{step-5} & q(a)
        & & \ans{ \Cref{step-3,step-4} and the Rule of Disjunctive Syllogism }\\
    \ITEM \label{step-6} & \forall\,x\ [\neg q(x) \vee r(x)]
        & & \ans{ Premisse }\\
    \ITEM \label{step-7}& \neg q(a) \vee r(a)
        & & \ans{\Cref{step-6} and the Rule of Universal Specification }\\
    \ITEM \label{step-8} & q(a) \to r(a)
        & & \ans{ \Cref{step-7} and the rule of Material Implication ($P\to Q \Leftrightarrow \neg P \vee Q$). } \\
    \ITEM \label{step-9} & r(a)
        & & \ans{ Modus Ponens on \Cref{step-5,step-8}.}\\
    \ITEM \label{step-10} & \forall\,x\ [s(x) \to \neg r(x)]
        & & \ans{Premisse} \\
    \ITEM \label{step-11} & s(a) \to \neg r(a)
        & & \ans{ \Cref{step-10} and the Rule of Universal Specification}\\
    \ITEM \label{step-12} & r(a) \to \neg s(a)
        & & \ans{Transposition ($P\to Q \Leftrightarrow \neg Q \to \neg P$) and \cref{step-11}}\\
    \ITEM \label{step-13} & \neg s(a)
        & & \ans{Modus Tollens on \cref{step-9,step-12}}\\
    \ITEM \label{step-14} & \therefore \ \exists \, x \ \neg s(x) 
        & & \Cref{step-13} and the definition of the truth for $\exists\,x\ \neg s(x)$. The reason for this step is also referred to as the \emph{Rule of Existential Generalization}.
  \end{tabular}
\end{problem}
\end{document}