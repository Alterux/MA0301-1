\documentclass[a4paper, english, 12pt]{article} % norsk & english is supported

\newcommand{\exerciseNumber}{1}
\newcommand{\solutions}{true} % Change to   true   to create solutions
% Must be defined before course package, \exerciseNumber, and \solutions
\usepackage{IMF}

\usepackage{MA0301}


\begin{document}

\titlebox

\seksjon{2.1}

\begin{problem}[3]
  Let $p$, $q$ be the primitive statements for which the implication $p \to q$
  false. Determine the truth values for each of the following:
\end{problem}

\begin{subproblem}[2]
  $\neg p \vee q$
\end{subproblem}

\begin{answer}
  The implication $p \to q$ is only false when $p$ is True and $q$ is
  False. Thus, $\neg p \vee q$ is False (or $0$).
\end{answer}

\begin{subproblem}[4]
  $\neg q \to \neg p$
\end{subproblem}

\begin{answer}
  This is just the same staten written as a contrapositive. $\neg q$ = True and
  $\neg p$ = False. Thus, the statement is False (or $0$).
\end{answer}

\begin{problem}[8]
  Construct a truth table for each of the following compounded statements, where
  $p$, $q$, $r$ denote primitive statements.
\end{problem}

\begin{subproblem}
  $(p \to q) \to (q \to p)$
\end{subproblem}

\begin{answer}
  \begin{tabular}{L L L L L}
    \toprule
    p & q & (p \to q) & (q \to p) & (p \to q) \to (q \to p) \\
    \midrule
    \F & \F & \T & \T & \T \\
    \F & \T & \T & \F & \F \\
    \T & \F & \F & \T & \T \\
    \T & \T & \T & \T & \T \\
    \bottomrule
  \end{tabular}
\end{answer}
%

\begin{subproblem}
  $[p \wedge (p \to q)] \to q$ 
\end{subproblem}

\begin{answer}
  \begin{tabular}{L L L L L}
    \toprule
    p & q & (p \to q) & [p\wedge(p\to q)] & [p \wedge (p \to q)] \to q \\
    \midrule
    \F & \F & \T & \F & \T \\
    \F & \T & \T & \F & \T \\
    \T & \F & \F & \F & \T \\
    \T & \T & \T & \T & \T \\
    \bottomrule
  \end{tabular}
\end{answer}

\begin{subproblem}
  $(p \wedge q) \to p$ 
\end{subproblem}

\begin{answer}
  \begin{tabular}{L L L L}
    \toprule
    p & q & (p \wedge q) & (p \wedge q) \to p\\
    \midrule
    \F & \F & \F & \T \\
    \F & \T & \F & \T \\
    \T & \F & \F & \T \\
    \T & \T & \T & \T \\
    \bottomrule
  \end{tabular}
\end{answer}
  
\begin{subproblem}
  $q \leftrightarrow (\neg p \vee \neg q)$ 
\end{subproblem}

\begin{answer}
  \begin{tabular}{L L L L}
    \toprule
    p & q & (\neg p \vee \neg q) & q \leftrightarrow (\neg p \vee \neg q) \\
    \midrule
    \F & \F & \T & \F \\
    \F & \T & \T & \F \\
    \T & \F & \T & \F \\
    \T & \T & \F & \F \\
    \bottomrule
  \end{tabular}
\end{answer}
  
\begin{subproblem}
  $[(p \to q) \wedge (q \to r)] \to (p \to r)$
\end{subproblem}

\begin{answer}
  \begin{tabular}{L L L L L L L}
    \toprule
    p & q & r & (p \to q) & (p \to r) & (q \to r) & [(p \to q) \wedge (q \to r)] \to (p \to r)\\
    \midrule
    \F & \F & \F & \T & \T & \T & \F \\
    \F & \F & \T & \T & \T & \T & \T \\
    \F & \T & \F & \T & \T & \F & \T \\
    \F & \T & \T & \T & \T & \T & \T \\
    \T & \F & \F & \F & \F & \T & \T \\
    \T & \F & \T & \F & \T & \F & \T \\
    \T & \T & \F & \T & \F & \T & \F \\
    \T & \T & \T & \T & \T & \T & \T \\
    \bottomrule
  \end{tabular}
\end{answer}
  
\begin{problem}[15]
  The integer variables $m$ and $n$ are assigned the values $3$ and $8$,
  respectively, during the execution of a program. Each of the following
  \emph{successive} statements is then encountered during program execution.
  What are the values of $m$, $n$ after each of these statements are
  encountered?
\end{problem}
  
\begin{subproblem}
  \begin{algorithmic}
    \STATE \IF{$n - m = 5$} \STATE $n := n - 2$ \ENDIF
  \end{algorithmic}
\end{subproblem}

\begin{answer}
  $m = 3$ and $n = 8  - 2 = 6$
\end{answer}

\begin{subproblem}
  \begin{algorithmic}
    \STATE \IF{$(2*m = n)$ \AND $(\lfloor n/4 \rfloor = 1)$}
    \STATE $n := 4 * m - 3$ \ENDIF
  \end{algorithmic}
\end{subproblem}

\begin{answer}
    $m = 3$ and $n = 4\cdot 3 - 3 = 9$.
\end{answer}
    
\begin{subproblem}
  \begin{algorithmic}
    \STATE \IF{$(n < 8)$ \OR $(\lfloor m/2 \rfloor = 3)$}
    \STATE $n := 4 * m - 3$
    \ELSE
    \STATE $m = 2*n$
    \ENDIF
  \end{algorithmic}
\end{subproblem}

\begin{answer}
  Neither of the conditions are true. Thus,  $m = 2\cdot 9 = 18$ and $n=9$.
\end{answer}
  
\begin{subproblem}
  \begin{algorithmic}
    \STATE
    \IF{$(m < 20)$ \AND $(\lfloor n/6 \rfloor = 1)$}
    \STATE $m := m - n - 5$
    \ENDIF
  \end{algorithmic}
\end{subproblem}

\begin{answer}
  Both conditions holds, thus $m = 18 - 9 - 5 = 4$ and $n = 9$.
\end{answer}

\begin{subproblem}
  \begin{algorithmic}
    \STATE \IF{$((n = 2 * m)$ \OR $(\lfloor n/2\rfloor = 5))$}
    \STATE $n := 4 * m - 3$ \ENDIF
  \end{algorithmic}
\end{subproblem}

\begin{answer}
  Neither conditions holds, thus $m = 4$ and $n = 9$.
\end{answer}

\seksjon{2.2}
  
\begin{problem}[6]
  Negate each of the following and simplify the resulting statement
\end{problem}

\begin{subproblem}[2]
  $(p \wedge q) \to r$
\end{subproblem}

\begin{answer}
\begin{align*}
  \neg [(p \wedge q) \to r]
  = p \wedge q \wedge \neg r 
\end{align*}
  %
  As $\neg(a \implies b) = a \wedge \neg b$ I am not sure how to expand on the calculations..
\end{answer}

\begin{subproblem}
  $p \vee q \vee (\neg p \wedge \neg q \wedge r)$
\end{subproblem}

\begin{answer}
\begin{align*}
  \neg \left[ p \vee q \vee (\neg p \wedge \neg q \wedge r)\right] 
  & = \neg p \wedge \neg q \wedge \neg (\neg p \wedge \neg q \wedge r) \\
  & = \neg p \wedge \neg q \wedge (p \vee q \vee \neg r)  \\
  & = \neg p \wedge \neg q \wedge \neg r
\end{align*}
  %
  Used DeMorgans laws in the first equality and removed inverses in the second.
\end{answer}


\begin{problem}[7]
  \begin{subproblem}
      If $p$, $q$ are primitive statements, prove that
      %
      \begin{equation*}
        (\neg p \vee q) \wedge (p \wedge (p \wedge q)) \ \leftrightarrow \ (p \wedge q)
      \end{equation*}
  \end{subproblem}%
\end{problem}

\begin{answer}
\begin{align*}
  (\neg p \vee q) \wedge (p \wedge (p \wedge q))
  \ \Leftrightarrow \ & (\neg p \vee q) \wedge ((p \wedge p) \wedge q) && \text{Commutative laws}\\ 
  \ \Leftrightarrow \ & (\neg p \vee q) \wedge (p \wedge q) && \text{Idempotent laws}\\ 
  \ \Leftrightarrow \ & [ (\neg p \wedge p) \vee (q \wedge p) ] \wedge q && \text{Distributive laws}\\
  \ \Leftrightarrow \ & [ F_0 \vee (q \wedge p) ] \wedge q && \text{Inverse laws}\\
  \ \Leftrightarrow \ & [ (q \wedge p) ] \wedge q && \text{Identity laws}\\
  \ \Leftrightarrow \ & p \wedge (q \wedge q)  && \text{Commutative laws}\\
  \ \Leftrightarrow \ &  (p \wedge q) && \text{Idempotent laws}\\
\end{align*}
% 
Alternatively the statement follows from the table below
 
\begin{tabular}{L L L L L L L}
  \toprule
  p & q & (\neg p \vee q) & (p \wedge (p \wedge q))
  &(\neg p \vee q) \wedge (p \wedge (p \wedge q)) & (p \wedge q)\\
  \midrule
  \F & \F & \T & \F & \F & \F \\
  \F & \T & \T & \F & \F & \F \\
  \T & \F & \F & \F & \F & \F \\
  \T & \T & \T & \T & \T & \T \\
  \bottomrule
\end{tabular}
\end{answer}

\begin{subproblem}
  Write the dual of the logical equivalence in \textbf{a)}.
\end{subproblem}

\begin{answer}
The dual of a compound proposition that contains only the logical operators $\vee$
, $\wedge$ , and $\neg$ is the compound proposition obtained by replacing each
$\vee$ by $\wedge$
, each $\wedge$ by $\vee$ , each $T_0$ by $F_0$ , and each $F_0$ by $T_0$
  %
\begin{equation*}
  (\neg p \wedge q) \vee (p \wedge (p \wedge q)) \leftrightarrow (p \vee q).
\end{equation*}
\end{answer}

\begin{problem}[18]
  Give the reasons for earch step in the following simplifications of compound
  statements
      \begin{flalign*}
        \textbf{\stepcounter{subCounter}\alph{subCounter})}
        && & [(p \vee q) \wedge (p \vee \neg q)] \vee q && \textbf{Reasons}\\
        && \Leftrightarrow \ & [p \vee (q \wedge \neg q)] \vee q &&
        \ans{\text{Idempotent \& Commutative \& Distributive laws}}\\
        && \Leftrightarrow \ & (p \vee F_0 ) \vee q && \ans{\text{Inverse Laws}}\\
        && \Leftrightarrow \ & p \vee q && \ans{\text{Identity laws}}
     \end{flalign*}%
\end{problem}

\end{document}