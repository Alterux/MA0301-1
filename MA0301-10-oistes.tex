\documentclass[a4paper, english, 12pt]{article} % norsk & english is supported

\newcommand{\exerciseNumber}{10}
\newcommand{\solutions}{true} % Change to   true   to create solutions
% Must be defined before course package, \exerciseNumber, and \solutions
\usepackage{IMF}

\usepackage{MA0301}

\usepackage{tikz}

\newcommand{\labeledRectangle}[9]{%
  \begin{tikzpicture}
    \begin{scope}[execute at begin node=$, execute at end node=$]
      \def\a{#1}
      \draw[] (0,0)   -- (\a,0) node[near start, below] {#7} node[near end, below] {#6};
      \draw[] (\a,0)   -- (\a,\a) node[near start, right] {#5} node[near end, right] {#4};
      \draw[] (\a,\a)   -- (0,\a) node[near start, above] {#3} node[near end, above] {#2};
      \draw[] (0,\a)   -- (0,0) node[near start, left] {#9} node[near end, left] {#8};
    \end{scope}%
  \end{tikzpicture}%
}

\begin{document}

\titlebox

\seksjon{1.1 \& 1.2}

\begin{problem}[11]
  Three small towns, designated by $A$, $B$, $C$ are interconnected by a system
  of two-way roads, as shown in Fig. 4
\end{problem}
%
\begin{figure}[h!tbp]
  \centering
  \begin{tikzpicture}
    \matrix (m) [matrix of math nodes,
    row sep=3em,
    column sep=3em,
    minimum width=2em,
    nodes in empty cells, 
    nodes={anchor=center}]
    {
      \node (A) {A}; &                    & \node (B) {B};    &                   & \node (C) {C}; \\
      \node (D) {};  &                    &                   &                   & \node (E) {};\\
    };
    \path[draw] (A.east) -- node [midway,below] {$R_2$} (B.west);
    \path[draw] (B.east) -- node [midway, below] {$R_6$} (C.west);

    \path[draw] (A.north east) edge[bend left =45] node [midway, above] {$R_1$} (B.north west);
    \path[draw] (A.south east) edge[bend right =45] node [midway, below] {$R_3$}(B.south west);
    \path[draw] (A.south) edge[in = -90, out = -90, looseness = 1.5] node [midway, below] {$R_4$}(B.south);
    
    \path[draw] (B.north east) edge[bend left =45] node [midway, above] {$R_5$}(C.north west);
    \path[draw] (B.south east) edge[bend right =45] node [midway, below] {$R_7$}(C.south west);

    \path[draw] (A.north) edge[in=90,out=90] node [midway, below] {$R_8$} (C.north);
    \path[draw] (A.south) edge [bend right = 45] (D.center)
    (D.center) edge[bend right = 45] node[midway, above] {$R_9$} (E.center)
    (E.center) edge [bend right = 40] (C);

  \end{tikzpicture} 
  \caption{}
\end{figure}

\begin{subproblem}
  In how many ways can Linda travel from town $A$ to town $C$?
\end{subproblem}

\begin{answer}
  To reach $C$ we can either go directly along $R_8$ or $R_9$, this gives $2$
  paths, or travel through $B$. In the last case there are $4$ paths to $B$, and
  for each of these there are $3$ paths from $B$ to $C$. In total
  %
  \begin{equation*}
    \text{\# paths from $A$ to $C$:} \quad 2 + 4 \cdot 3 = 14.
  \end{equation*}
\end{answer}

\begin{subproblem}
  \label{subproblem:11b}
  How many different round strips can Linda travel from town $A$ to town $C$ and
  back to town $A$?
\end{subproblem}

\begin{answer}
  From \cref{subproblem:11b} we know that there are $14$ paths from $A$ to $C$,
  and since every path is a two-way path, this means there are $14$ paths from
  $C$ to $A$. Thus, for each of the $14$ paths we have taken to reach $C$ we can
  choose $14$ different paths to return. 
  \begin{equation*}
    \text{\# paths from $A$ to $C$ and back to $C$:} \quad 14^2 = 196.
  \end{equation*}
\end{answer}

\begin{subproblem}
  How many of the round trips in \cref{subproblem:11b} are such that the return
  trip (from town $C$ to town $A$) is atleast partially different from the route
  Linda takes from town $A$ to town $C$?
\end{subproblem}

\begin{answer}
  There are $14$ paths from $A$ to $C$, thus once we have reached $C$ there are
  $13$ paths that we have \emph{not} taken. Thus, to obtain a partially
  different path, we just have to pick one of the $13$ paths not yet taken.
  %
  \begin{equation*}
    \text{\# round trips $A \to C \to A$ with different return path:} \quad 14 \cdot 13 = 182.
  \end{equation*}
\end{answer}

\begin{problem}[24]
  Show that for all integers $n, r \geq 0$, if $n + 1 \geq r$ then
  %
  \begin{equation*}
    P(n + 1, r) = \left( \frac{n + 1}{n + 1 - r} \right) P(n, r),
  \end{equation*}
  %
  where $P(n, r) = n!/(n - r)!$ denotes the number of permutations.
\end{problem}

\begin{answer}
  Using that $n! = n(n-1)!$ (Example: $5! = 5\cdot 4!$) gives us the relation directly
  %
  \begin{align*}
    P(n + 1, r)
    = \frac{(n+1)!}{(n+1-r)!}
    = \frac{n+1}{n+1-r} \cdot \frac{n!}{(n-r)!}
    = \frac{n+1}{n+1-r} P(n, r)
  \end{align*}
\end{answer}

\begin{problem}
  Find the values(s) of $n$ in each of the following:
  \begin{enumerate}[label = (\alph*)]
    \item $P(n, 2) = 90$
    \item $P(n, 3) = 3 P(n, 2)$
    \item $2 P(n, 2) + 50 = P(2n, 2)$
  \end{enumerate}
\end{problem}

\begin{answer}
  Note that for all $n \in \N$ we have
  %
  \begin{equation}
    \label{eq:P(n2)}
    P(n, 2) = \frac{n!}{(n - 2)!} = \frac{n(n-1)(n-2)!}{(n-2)!} = n(n-1)
  \end{equation}
  %
  As such we are looking for two integers, $n$, $n-1$ such that $n \cdot (n-1) =
  90$. By inspection we see that $90 = 10 \cdot 9 = 10 (10 - 1)$, so $n=10$. We
  also have that $90 = (-9)(-10) = -9(-9-1)$ so another solution is $n=-9$,
  however I do not think we are counting negative solutions when dealing with
  permutations.

  Using \cref{eq:P(n2)} we have
  %
  \begin{equation*}
    P(n, 2) = P(n, 3) \ \Leftrightarrow \ n(n-1) = n(n-1)(n-2)
                      \ \Leftrightarrow \ n(n-1)(3 - n) = 0
  \end{equation*}
  %
  So $n = 0$, $n = 1$ or $n = 3$.In the last step we simply factored out the
  common term,
  %
  \begin{equation*}
    n(n - 1) \cdot 1 - n(n-1) \cdot (n-2) = n(n-1) [1 - (n-2)] = n(n-1)(3 - n).
  \end{equation*}
  %
  Again we use the idea from \cref{eq:P(n2)} to rewrite the equation
  %
  \begin{equation*}
    2P(n, 2) + 50 = P(2n, 2) \ \Leftrightarrow \ 2 n(n-1) + 50 = 2n(2n-1)
                               \Leftrightarrow \ 2 n^2 = 50
  \end{equation*}
  %
  Since $ 50 = 2 \cdot 5^2$ we see by inspection that the solutions to the
  equation above is $n = \pm 5$. Where again I am quite sure that we are only
  interested in the positive solution.
\end{answer}

\begin{problem}[27]
  \begin{subproblem}[2]
    \label{subproblem:27b}
    How many distinct paths are there from $(1, 0, 5)$ to $(8, 1, 7)$ in
    Euclidian three-space if each more is one of the following types?
    %
    \begin{enumerate}
        \item[(H):] $(x, y, z) \to (x + 1, y, z)$:
        \item[(V):] $(x, y, z) \to (x, y + 1, z)$:
        \item[(A):] $(x, y, z) \to (x, y, z + 1)$
    \end{enumerate} 
  \end{subproblem}
\end{problem}

\begin{answer}
  Each path consists of $2$ H's, $1$ V and $7$ A's. There are
  %
  \begin{equation*}
    \frac{10!}{2! \cdot 1! \cdot 7!} = 360
  \end{equation*}
  %
  ways to arrange these $10$ letters, and this is the number of paths. 
\end{answer}

\begin{subproblem}
  Generalize the results in \cref{subproblem:27b}.
\end{subproblem}

\begin{answer}
  If $a$, $b$, and $c$ are any real numbers and $m$, $n$, $p$ are non-negative
  integers, then the number of paths from $(a, b, c)$ to $(a + m, b + n, c \ p)$
  is
  %
  \begin{equation*}
    \frac{(m+n+p)!}{m! \cdot n! \cdot p!}
  \end{equation*}
\end{answer}

\begin{problem}[36]
  \begin{subproblem}
    In how many ways can eight people, denoted $A$, $B$, $\ldots$, $H$ be seated
    about the square table shown in \cref{fig:tables}. Where \cref{fig:table-a,fig:table-b} are
    considered the same but are distinct from \cref{fig:table-c}?
  \end{subproblem}
\end{problem}

\begin{figure}[htbp!]
  \centering
  \begin{subfigure}[b]{0.3\textwidth}
    \centering
    \labeledRectangle{3}{A}{B}{C}{D}{E}{F}{G}{H}
    \caption{}
    \label{fig:table-a}
  \end{subfigure}
  ~ %add desired spacing between images, e. g. ~, \quad, \qquad, \hfill etc. 
  % (or a blank line to force the subfigure onto a new line)
  \begin{subfigure}[b]{0.3\textwidth}
    \centering
    \labeledRectangle{3}{G}{H}{A}{B}{C}{D}{E}{F}
    \caption{}
    \label{fig:table-b}
  \end{subfigure}
  ~ %add desired spacing between images, e. g. ~, \quad, \qquad, \hfill etc. 
  % (or a blank line to force the subfigure onto a new line)
  \begin{subfigure}[b]{0.3\textwidth}
    \centering
    \labeledRectangle{3}{F}{G}{H}{A}{B}{C}{D}{E}
    \caption{}
    \label{fig:table-c}
  \end{subfigure}
  \caption{}\label{fig:tables}
\end{figure}

\begin{answer}
  The exclusion of arrangements that can be obtained from rotation comes to the
  same as the extra condition that $A$ is seated e.g. at the upper side. This
  because in any case there is exactly one rotation that brings him there. Then
  there are $2$ possibilities for $A$.  Now $B$ has $7$ possible seats
  to choose from, $C$ has $6$ possible seats to choose from and so forth. In
  total we have
  %
  \begin{equation*}
    \text{\# Distinct table seatings for 8 people: } 2 \cdot 7! = 7\cdot 6 \cdots 2\cdot 2 = 10080
  \end{equation*}
  %
  Another way to obtain this answer is as follows: Ignoring symmetries and
  rotations there are $8!$ ways to place $8$ persons around the table. If we
  reflect the table around the $x$-axis, or $y$-axis we obtain the same table
  placement. These are shown in \cref{fig:rotations}. So for every table placement, we have
  $4$ identitcal positions. Thus, $8!/4 = 8\cdot 7!/4 = 2 \cdot 7!$ as before.
\end{answer}

\begin{Figure}[htbp!]
  \centering
  \begin{subfigure}[b]{0.25\textwidth}
    \centering
    \labeledRectangle{2}{A}{B}{C}{D}{E}{F}{G}{H}
    \caption{No rotation}
    \label{fig:no-rotation}
  \end{subfigure}
  ~ %add desired spacing between images, e. g. ~, \quad, \qquad, \hfill etc. 
  % (or a blank line to force the subfigure onto a new line)
  \begin{subfigure}[b]{0.25\textwidth}
    \centering
    \labeledRectangle{2}{F}{E}{D}{C}{B}{A}{H}{G}
    \caption{Vertical}
    \label{fig:x-rotation}
  \end{subfigure}
  
  \begin{subfigure}[b]{0.25\textwidth}
    \centering
    \labeledRectangle{2}{B}{A}{H}{G}{F}{E}{D}{C}
    \caption{Horizontal}
    \label{fig:y-rotation}
  \end{subfigure}
  ~ %add desired spacing between images, e. g. ~, \quad, \qquad, \hfill etc. 
  % (or a blank line to force the subfigure onto a new line)
  \begin{subfigure}[b]{0.25\textwidth}
    \centering
    \labeledRectangle{2}{E}{F}{G}{H}{A}{B}{C}{D}
    \caption{Horizontal \& vertical}
    \label{fig:xy-rotation}
  \end{subfigure}
  \caption{The $4$ rotation symmetries}\label{fig:rotations}
\end{Figure}


\begin{subproblem}
  If two of the eight people, say $A$ and $B$, do not get along well, how many
  different seattings are possible with $A$ and $B$ not sitting next to each other?
\end{subproblem}

\begin{answer}
  We begin as before by placing $A$. To avoid symmetries $A$ is once again
  placed on the upper half of the table, and as such have $2$ seats to choose
  from. Now as $B$ does not want to sit next to $A$ he only has $5$ seats to
  choose from, while $C$ still has $6$. Continuing we get that the total number
  of arrangements such that $A$ does not sit beside $B$ are
  \begin{equation*}
    \text{\# Arrangements such that $A$ does not sit beside $B$: } \quad
    2 \cdot 5 \cdot 6 \cdot 5 \cdot 4 \cdot 3 \cdot 2 = 7200
  \end{equation*}
\end{answer}


\seksjon{1.3}

\begin{problem}[16]
  Determine the value of each of the following summations,
\end{problem}

\begin{subproblem}[3]
  $\displaystyle \sum_{i = 0}^{10} 1 + (-1)^i \ans{= (1 - 1) + (1 + 1) + \cdots
    + (1 +
  1) = 6 \cdot 2 = 12}$
\end{subproblem}

\begin{answer}
  Notice in general that for every \emph{odd} $i = 2k+1$  then $1 + (-1)^{2k+1}
  = 1 - 1 = 0$, and similarly for \emph{even} $i = 2k$ then $1 + (-1)^{2k} = 2$.
  Let $n \in \N$ then in general we have
  %
  \begin{align*}
    \sum_{i = 0}^{2n} 1 + (-1)^i
    & = (1 - 1) + (1 + 1) + \cdots + (1 + 1) \\
    & = \sum_{i = 0}^n (1 + (-1)^{2k-1}) + (1 + (-1)^{2k})
    = \sum_{i = 0}^n (1 - 1) + (1 + 1)
    = 2 (n + 1)
  \end{align*}
\end{answer}

\begin{subproblem}
  $\displaystyle \sum_{k = n}^{2n} (-1)^k$ where $n$ is an odd positive integer
\end{subproblem}

\begin{answer}
  Notice that when $k$ is odd we have $(-1)^k = - 1$ and $(-1)^{k} = 1$ when $k$
  is even. Thus, every other term cancel each other out. Since $n$ is odd we
  will always sum an even number of terms, so the sum is zero.
  %
  \begin{align*}
    \sum_{k=n}^{2n} (-1)^k
    & = (-1)^{n} + (-1)^{n+1} + \cdots + (-1)^{2n-1} + (-1)^{2n} \\
    & = (-1 + 1) + (-1 + 1) + \cdots + (-1 + 1)
      = 0
  \end{align*}
  %
  the first term $(-1)^n$ is negative because $n$ is \emph{odd}.
\end{answer}

\begin{subproblem}
  $\displaystyle \sum_{i = 1}^{6} i(-1)^i \ans{= 1(-1)^1 + 2(-1)^2 + \cdots + 
    6(-1)^6 = -1 + 2 - 3 + 4 - 5 + 6 = 3}$
\end{subproblem}

\begin{answer}
  Let $n \in \N$ then in general we have
  %
  \begin{align*}
         \sum_{i = 1}^{2n} i(-1)^i
     = & (-1 + 2) + (-3 + 4) + \cdots + \bigl[  (2k - 1) (-1)^{2k-1} + 2k(-1)^{2k}\bigr] \\
     = & \sum_{k = 1}^{n} \bigr[ (2k-1)(-1)^{2k-1} + 2k(-1)^{2k}\bigl] 
     =   \sum_{k = 1}^{n} \bigr[ (2k - 1)(-1) + 2k \bigl] 
      =  \sum_{k = 1}^n 1
      = n
  \end{align*}
\end{answer}

\begin{problem}[25]
  Determine the coefficient of
  \begin{enumerate}[label = \textbf{\alph*)} ]
    \item $xyz^2$ in $(x + y + z)^4$
    \item $xyz^2$ in $(w + x + y+ z)^4$
    \item $xyz^2$ in $(2x - y - z)^4$
    \item $zyz^{-2}$ in $(x - 2y + 3z^{-1})^4$
    \item $w^3x^2yz^2$ in $(2w - x + 3y - 2z)^{8}$
  \end{enumerate} 
\end{problem}

\begin{answer}
  For any positive integer $m$ and any nonnegative integer $n$, the multinomial
  formula tells us how a sum with $m$ terms expands when raised to an arbitrary
  power $n$:
  % 
  \begin{equation*}
    (x_1 + x_2  + \cdots + x_m)^n 
    = \sum_{k_1+k_2+\cdots+k_m=n} {n \choose k_1, k_2, \ldots, k_m}
    \prod_{t=1}^m x_t^{k_t}\,,
  \end{equation*}
  where
  % 
  \begin{equation*}
    \binom{n}{k_1, k_2, \ldots, k_m} = \frac{n!}{k_1!\, k_2! \cdots k_m!}
  \end{equation*}
  %
  is a multinomial coefficient. The sum is taken over all combinations of
  nonnegative integer indices $k_1$ through $k_m$ such that the sum of all $k_i$ is $n$.
  That is, for each term in the expansion, the exponents of the $x_i$ must add up
  to $n$.

  \begin{enumerate}[label =\textbf{\alph*)}]
    \item $xyz^2$ in $(x+y+z)^4$ has the coefficient $\displaystyle
      \binom{4}{1,1,2} = \frac{4!}{1! \cdot 1! \cdot 2!} = 12$.
    \item $xyz^2$ in $(w + x + y+ z)^4$ has the coefficient $\displaystyle
      \binom{4}{0, 1, 1, 2} = \frac{4!}{0!\cdot 1! \cdot 1! \cdot 2!} = 12$.
    \item $xyz^2$ in $(2x - y - z)^4$ has the coefficient $\displaystyle
      \binom{4}{1,1,2} (2)(-1)(-1)^2= -\frac{4! \cdot 2}{1! \cdot 1! \cdot 2!} = -24$.
    \item $zyz^{-2}$ in $(x - 2y + 3z^{-1})^4$ has the coefficient $\displaystyle
      \binom{4}{1,1,2}(-2)(3)^2 = -\frac{4! \cdot 18}{1! \cdot 1! \cdot 2!} = 216$.
    \item $w^3x^2yz^2$ in $(2w - x + 3y - 2z)^{8}$ has the coefficient $\displaystyle
      \binom{8}{3,2,1,2} (2)^3 (-1)^2 (3)(-2)^2= \num{161280}$.
  \end{enumerate}
\end{answer}

\newpageanswer

\begin{problem}[33]
  \begin{subproblem}[2]
    \label{subproblem:33b}
    Given a list $a_0, a_1, a_2, \ldots, a_n$ --- of $n+1$ real numbers, where
    $n$ is a positive integer, determine
    % 
    \begin{equation*}
      \sum_{i = 1}^{n} a_i - a_{i - 1}\,.
    \end{equation*}
  \end{subproblem}
\end{problem}

\begin{answer}
  A series of this form is known as a
  \href{https://en.wikipedia.org/wiki/Telescoping_series}{telescoping series}.
  In mathematics, a \emph{telescoping series} is a series whose partial sums
  eventually only have a fixed number of terms after cancellation. The
  cancellation technique, with part of each term cancelling with part of the
  next term, is known as the \emph{method of differences}.

  In particular for the series above
  %
  \begin{align*}
    \sum_{i = 1}^{n} a_i - a_{i - 1}
    & = (a_1 - a_0) + (a_2 - a_1) + \cdots + (a_{n-1} + a_{n-2}) + (a_{n} - a_{n-1}) \\
    & = - a_0 + (a_1 - a_1) + (a_2 - a_ 2) + \cdots (a_{n-1} - a_{n-1}) + a_n  \\
    & = a_n - a_0
  \end{align*}
\end{answer}

\begin{subproblem}
  Determine the value of $ \ \ \displaystyle \sum_{i = 1}^{100} \frac{1}{i + 2} -
  \frac{1}{i + 1}$.
\end{subproblem}

\begin{answer}
  \noindent
  This is a telescoping series with $a_i = 1/(i + 2)$. Using the result from
  \cref{subproblem:33b} we obtain
  %
  \begin{equation*}
        \sum_{i = 1}^{100} \frac{1}{i+2} - \frac{1}{i+1}
    =   \frac{1}{100 + 2} - \frac{1}{0 + 2}
    = - \frac{25}{51}.
  \end{equation*}
\end{answer}


\seksjon{5.6}

\begin{problem}[8]
  Let $f \colon A \to B$, $g \colon B \to C$. Prove that
  \begin{enumerate}[label = (\alph*)]
    \item if $g \circ f \colon A \to C$ is onto, then $g$ is onto;
    \item if $g \circ f \colon A \to C$ is one-to-one, then $f$ is one-to-one.
  \end{enumerate}
\end{problem}

\begin{answer}
  \begin{enumerate}[label = (\alph*)]
    \item Assume that $g \circ f \colon A \to C$ is onto. The function $g \circ
      f$ from the set $A$ to a set $C$ is onto, if for every element $y$ in the
      codomain $C$ of $f$ there is at least one element $x$ in the domain $A$
      such that $(g \circ f)(x) = y.$

      In other words, if $y \in C$, there exists some element $x \in A$ such that
      $(g \circ f)(x) = y$. Then $g(f(x)) = y$ with $f(x) \in B$, so $g$ is
      onto.

      \item Let $x, y \in A$ if $(g \circ f)(x) = (g \circ f)(y)$ implies that $x = y$ for
      all pairs $x,y$ then $g \circ f$ is one-to-one. Now,
      %
      \begin{align*}
        f(x) = f(y)
        \ \Rightarrow \ g(f(x)) = g(f(y))
        \ \Rightarrow \ (g \circ f)(x) = (g \circ f)(y)
        \ \Rightarrow \ x = y,
      \end{align*}
      %
      since $g \circ f$ is one-to-one. Which is what we wanted to prove.
  \end{enumerate}
\end{answer}

\begin{problem}[17]
  Let $f, g \colon \Z^+ \to \Z^+$ where for all $x \in \Z^+$, $f(x) = x + 1$ and
  $g(x) = \max\{1, x - 1\}$, the maximum of $1$ and $x - 1$. 
\end{problem}

\begin{subproblem}
  What is the range of $f$?
\end{subproblem} 

\begin{answer}
  The range of a function $f$ is by definition
  \begin{equation*}
    \{y \mid \text{there exists an $x$ in the domain of $f$ such that} \ y = f(x)\}.
  \end{equation*}
  Since $x \in \Z^+$ and $f(x) = x + 1$, the range of $f$ is $\Z^+ / \{-1\} = \{2, 3, 4, \ldots\}$.
\end{answer}

\begin{answer}
\end{answer}

\begin{subproblem}
  \label{subproblem:17b}
  Is $f$ an onto function?
\end{subproblem}

\begin{answer}
  A function $f$ from a set $\Z^+$ to a set $\Z^+$ is surjective (or onto), or a
  surjection, if for every element $y$ in the codomain $\Z^+$ of $f$ there is at
  least one element $x$ in the domain $\Z^+$ of $f$ such that $f(x) = y$.

  \textbf{No.} Let $y = 1$, then there is no $x \in \Z$ such that $y = f(x)$. 
\end{answer}

\begin{answer}
\end{answer}

\begin{subproblem}
  Is the function $f$ one-to-one?
\end{subproblem} 

\begin{answer}
  An injective function or injection or one-to-one function is a function that
  preserves distinctness: it never maps distinct elements of its domain to the
  same element of its codomain. In other words, every element of the function's
  codomain is the image of at most one element of its domain.

  \textbf{Yes.} Let $x, y \in \Z^+$ if $f(x) = f(y)$ implies that $x = y$ for
  all pairs $x,y$ then $f$ is one-to-one. This is equivalent with the definition
  above
  %
  \begin{equation*}
    f(x) = f(y) \ \Rightarrow \ x + 1 = y + 1 \ \Rightarrow \ x = y
  \end{equation*}
\end{answer}

\begin{subproblem}
  What is the range of $g$?
\end{subproblem}

\begin{answer}
  Another equivalent way of writing the definition of $g$ is as follows:
  %
  \begin{equation*}
    g(x) =
    \begin{cases}
      \phantom{x - }\;\,1 & \text{if} \ x = 0 \\
      x - 1 & \text{otherwise} \\
    \end{cases}
  \end{equation*}
  Perhaps it is easier to see from this definition that the range of $g$ is $\Z_+$.
\end{answer}

\begin{subproblem}
  Is $g$ an onto function?
\end{subproblem} 

\begin{answer}
  As the domain and codomain are the same, the function is onto. For every $y
  \in \Z^+$ we can find a $x \in \Z^+$ such that $y = g(x)$.
\end{answer}

\begin{subproblem}
  Is the function $g$ one-to-one?
\end{subproblem} 

\begin{answer}
  \textbf{No.} Let $x, y \in \Z^+$ if $f(x) = f(y)$ implies that $x = y$ for
  all pairs $x,y$ then $f$ is one-to-one. However as $f(1) = f(2) = 1$, $f$
  is not one-to-one.
\end{answer}

\begin{subproblem}
  \label{subproblem:17g}
  Show that $g \circ f = 1_{\Z^+}$.
\end{subproblem}

\begin{answer}
  % 
  Using the definition above of $g$ we have that
  % 
  \begin{equation*}
    g(f(x)) = \max\{1 ,f(x) - 1\} = \max\{1, (x+1)-1\} = \max\{1,x\} = x,
  \end{equation*}
  % 
  since $x \in \Z^+$. This proves that $g \circ f = 1_{Z^+}$ as wanted.
\end{answer}

\begin{subproblem}
  \label{subproblem:17h}
  Determine $(f \circ g)(x)$ for $x = 2, 3, 4, 7, 12$ and $25$.
\end{subproblem} 

\begin{answer}
  We have $(f \circ g)(x) = 1 + \max\{1, x - 1\}$. This gives the following table
\end{answer}

\begin{table}[htbp!]
  \centering
  \caption{Shows $f(g(x)) = 1 + \max\{1, x - 1\}$ for various values.}
  \begin{tabular}{R | *{7}{R}}
    \toprule
    x \phantom{1.}  & 2 & 3 & 4 & 7 & 12 & 25 \\
    \midrule
    f(g(x)) & 2 & 3 & 4 & 7 & 12 & 25 \\
    \bottomrule
  \end{tabular}
\end{table}

\begin{subproblem}
  Do the answers for \cref{subproblem:17b,subproblem:17g,subproblem:17h} contradict the result in
  \hyperlink{thm:8}{Theorem 8}?
\end{subproblem} 

\begin{answer}
  \textbf{No.} The functions $f$, $g$ are \emph{not} inverses of each other. The
  calculations in \cref{subproblem:17h} may suggest that $f \circ g = 1_{Z^+}$
  since $(f \circ g)(x) = x$ for $x \geq 2$. But we also find that
  %
  \begin{equation*}
    (f \circ g)(1) = f(\max{1, 0}) = f(1) = 2,
  \end{equation*}
  %
  so $(f \circ)(1) \neq 1$, and, consequently, $f \circ g \neq 1_{Z^+}$.
\end{answer}

\noindent
\\\textbf{\hypertarget{thm:8}{Theorem 8:}} \textit{A function $f \colon A \to B$ is invertible if and only if it is
  one-to-one and onto.}

\end{document}
